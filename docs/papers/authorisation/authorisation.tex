% Vino Crescini  <jcrescin@cit.uws.edu.au>

\documentclass[draft]{article}
\author{Vino Fernando Crescini\\jcrescin@cit.uws.edu.au\\\\Intelligent Systems Laboratory\\School of Computing and Information Technology\\University of Western Sydney}
\title{Implementing a High-Level Description Language for a Sequence of Transformations of Authorisation Policies}
\date{17/03/2003}
\begin{document}
  \maketitle
  \section{Introduction}

    Authorisation policies are a collection of rules and constraints that 
    limits accesses to different resources within a system. Such policies may 
    be as simple as assigning read, write or execute permissions on specific 
    files, but some policies can be as high-level and abstract as to reflect 
    the organizational hierarchy of the system. For example, high-level 
    authorisation policies might include the following:

    \begin{itemize}
      \item Regional Managers have read and write access to the regional database.
      \item Branch Managers have no read and write access to the regional database.
      \item Branch Managers have read and write access to their branch database.
      \item If the branch manager is also a regional manager then he has read but not write access to the regional database.
    \end{itemize}

    \subsection{Background}

      Typically, in an access control system, authorisation policies are stored
      into a policy database or policy base. Another entity, the authorisation
      agent (or enforcer agent) ensures that the policies in the policy base
      are enforced. 

      At its simplest form, a policy base may be implemented as an access
      control matrix where the rows represent the subjects, the columns
      represent the objects or resources and each cell contains the access
      permissions. However, the access control matrix approach is limited in
      such a way that it lacks scalability and flexibility. The size of the
      access control matrix is directly proportional to the size of the objects
      times the number of subjects. It is not difficult to see that as the
      system gets larger, any authorisation policy mechanism that uses the
      access control matrix will get progressively slower and less efficient.

      A much better approach is the logic-based system where instead of
      storing each individual rules into a database, a generalized set of
      facts and constraints are stored into the policy base. There have been
      quite a few developments in this field. In their paper, Woo and Lam [3]
      describes a logic-based specification for policy bases. Bai and
      Varadharajan [1] went a few steps further by developing a high-level
      language for the transformation of the policy base from one state to
      another. A transformation, in this sense, is the process of adding,
      removing or modifying a rule in the policy base. Their approach also
      handles conflict resolution within a transformation. In another paper,
      Bai and Varadharajan [2] developed another language specification to
      handle sequences of transformations.

    \subsection{Problem}

    \subsection{Plan}

    \subsection{Organisation of this Paper}

    \pagebreak

  \section{A Formal Language}

    \subsection{Definition}

    \subsection{Syntax}

    \subsection{Semantics}

    \pagebreak

  \section{Translation into a Logic Program}

    \subsection{Description}

    \subsection{Examples}

    \pagebreak

  \section{System Structure}

    \pagebreak

  \section{Conclusion}

    \pagebreak

  \section{References}

\end{document}
