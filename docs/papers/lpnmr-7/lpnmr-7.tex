\documentclass{llncs}

\pagestyle{plain}

\begin{document}

  \long\def\comment#1{}

  \title{The Implementation of Policy Updater - A System of Dynamic Access Control}

  \author{\em Vino Fernando Crescini and Yan Zhang}

  \institute{School of Computing and Information Technology\\
             University of Western Sydney\\
             Penrith South DC, NSW 1797, Australia\\
             E-mail: \{jcrescin,yan\}@cit.uws.edu.au}

  \maketitle

  \begin{abstract}
  \end{abstract}

  \section{Introduction}

  \section{Language L}

    \subsection{Initial State Definition}

    \subsection{Constraint Definition}

    \subsection{Policy Manipulation}

      \subsubsection{Definition}

      \subsubsection{Directives}

    \subsection{Query Directive}

  \section{Implementation}

    \subsection{Translation to Logic Program}

      \subsubsection{L to Extended Logic Program}

        \begin{itemize}
          \item Integration of Set Theory
          \item State Handling Mechanism
        \end{itemize}

      \subsubsection{Extended Logic Program to Normal Logic Program}

        \begin{itemize}
          \item Removal of Classical Negation
        \end{itemize}

  \section{Application: Dynamic Access Control System in Web Servers}

  \begin{thebibliography}{5}

    \bibitem{yan:plp}
      Zhang and N.Y. Foo,
      Answer sets for prioritized logic programs.
      In {\em Proceedings of the 1997 International Logic Programming
      Symposium (ILPS'97)},
      pp 69-83. 
      MIT Press, 1997

  \end{thebibliography}

\end{document}

